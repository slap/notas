\documentclass[12pt]{article}
%\include{amsfonts}
\usepackage{amssymb,amsmath,amsthm,latexsym,epsfig,euscript,multicol}
%\usepackage{amsthm}
%\usepackage{enumitem}
\usepackage[utf8]{inputenc}
\usepackage[spanish]{babel}

% Caracteres especiales
\def\A{\mathbb{A}}
\def\C{\mathbb{C}}
\def \N{\mathbb{N}}
\def \P{\mathbb{P}}
\def \Q{\mathbb{Q}}
\def \R{\mathbb{R}}
\def \Z{\mathbb{Z}}
\def \sen{\textrm{sen}}

%\newtheorem{theorem}{Theorem}
%\newtheorem{prop}{Proposici\'on}[theorem]
%\newtheorem{lemma}[theorem]{Lemma}

%\theoremstyle{definition}
%\newtheorem{ejer}{Ejercicio}
\newcommand{\bej}{\begin{ejer}}
\newcommand{\fej}{\end{ejer}}

\def\dt{\Delta t}
\def\dx{\Delta x}

\topmargin-2cm \vsize 29.5cm \hsize 21cm
\setlength{\textwidth}{16.75cm}\setlength{\textheight}{23.5cm}
\setlength{\oddsidemargin}{0.0cm}
\setlength{\evensidemargin}{0.0cm}


\begin{document}

\newtheorem{theorem}{Theorem}
\newtheorem{prop}[theorem]{Proposici\'on}
%----------------------------------------------------------
%Encabezado:

\vskip 0.2cm

\date{\today}

 \vskip 0.2cm

\noindent
 {\bf Peque\~nos comentarios al apunte ``Elementos de C\'alculo Num\'erico'' de R. Dur\'an, S. Lasalle y J. Rossi}

 \makeatletter


\vskip 0.2cm

 \bigskip
 \centerline{\ttfamily \@date}
 \bigskip


\begin{itemize}

\item (p\'agina 25) En la demostraci\'on de $\|A\|_2 = \sqrt{\rho(A^TA)}$ para una matriz $A$ arbitraria, se utiliza que $A^TA$ es positiva semidefinida (y por lo tanto, todos los autovalores son no-negativos) para probar la igualdad tomando $\mu_j = \mu_{\max}$.
Esto se puede demostrar f\'acilmente por la igualdad $z^T A^T A z = (Az)^T Az = \|Az\|_2^2 \ge 0$.

\item (p\'agina 40, punto 1.) Tanto el algoritmo de Cholesky como el de Gauss tienen complejidad $O(N^3)$, pero la cantidad de productos de los algoritmos com\'unmente utilizados es $N^3/6$ para Cholesky y $N^3/3$ para Gauss. 

\item (p\'agina 44, demostraci\'on del Teorema 3.4). En la demostraci\'on de la desigualdad $\rho(A) \le \|A\|$, la afirmaci\'on $\|A\|$ en $\R$ es igual a $\|A\|$ en $\C$ vale para las normas comunes $\|\ \|_1$, $\|\ \|_2$, $\|\ \|_\infty$, usando las extensiones usuales a $\C$, pero una norma inducida en $\R$ puede no estar definida para elementos en $\C$, y se necesita una demostraci\'on m\'as general.

A continuaci\'on se da la demostraci\'on de esa desigualdad tomada de ``Matrices: Theory and Applications'', de Denis Serre (Proposici\'on 4.1.6, p\'agina 66):

\begin{prop}
 Para cualquier norma inducida $\|\ \|$ en $\R^{n\times n}$, se cumple $\rho(A) \le \|A\|$.
\end{prop}
\begin{proof}
 El caso $K = \C$ es simple, utilizando el argumento en el apunte.

 Para el caso $K = \R$, fijamos una norma inducida $N_\C$ en $\C^{n\times n}$ y sea $N_\R$ la restricci\'on a $\R^{n \times n}$. Es f\'acil ver que $N_\R$ es una norma en $\R^{n \times n}$ (verifica los axiomas de norma), aunque no necesariamente es una norma inducida.

 Dada ahora una norma inducida $\|\ \|$ en $\R^{n \times n}$, tenemos por la equivalencia de normas en cualquier espacio vectorial de dimensi\'on finita que
  \[
  \rho(A)^k = \rho(A^k) \le N_\C(A^k) = N_\R(A^k) \le C \|A^k\| \le C \| A\|^k,
 \]
 para alguna constante $C > 0$. Por lo tanto,
\[
 \rho(A) \le C^{1/k} \|A\|
\]
y tomando l\'imite $k \rightarrow \infty$ de ambos lados, obtenemos $\rho(A) \le \|A\|$.


\end{proof}


\item (p\'agina 47, demostraci\'on del Corolario 3.7) Es sencillo pero no es inmediato ver que si $\|B^k\|^{1/k} \rightarrow \rho(B)$ vale para una norma particular, vale para cualquier norma por equivalencia de normas. 
    
    Dada una norma $\| \|$, existen $C_1$ y $C_2$ tales que
    \[
    C_1 \|B^k\|_\infty \le \|B^k\| \le C_2 \|B^k\|_\infty
    \]
    y por lo tanto
    \[
    C_1^{1/k} \|B^k\|_\infty^{1/k} \le \|B^k\|^{1/k} \le C_2^{1/k} \|B^k\|_\infty^{1/k}.
    \]

    Tomando l\'imites,
    \[
    \lim_{k \rightarrow \infty} C_1^{1/k} \|B^k\|_\infty^{1/k} \le \lim_{k \rightarrow \infty} \|B^k\|^{1/k} \le \lim_{k \rightarrow \infty} C_2^{1/k} \|B^k\|_\infty^{1/k},
    \]
    y como $C_1^{1/k} \rightarrow 1$ y $C_2^{1/k} \rightarrow 1$ se obtiene

    \[
    \rho(B) \le \lim_{k \rightarrow \infty} \|B^k\|^{1/k} \le \rho(B)
    \]
    y por lo tanto $\lim_{k \rightarrow \infty} \|B^k\|^{1/k} = \rho(B)$.

    \item (p\'agina 72, segundo p\'arrafo) Se est\'a utilizando el desarrollo de Taylor de orden 1 (con t\'ermino de error) centrado en $x_n$.
    \end{itemize}

\end{document}
%---------------------------------------------------------
